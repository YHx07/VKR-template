\documentclass[a4paper, 12pt]{article}
\usepackage{style}
\usepackage{fancyhdr}
\usepackage[backend=biber, sorting=none]{biblatex}
\usepackage{csquotes}
\usepackage{fancyhdr}
\usepackage{titlesec}

\fancypagestyle{bib}{
	\fancyhf{}
	\rhead{\scshape\it{{СПИСОК ЛИТЕРАТУРЫ}}}
	\fancyfoot[C]{\thepage}
	\renewcommand{\headrulewidth}{0.4pt}
}

\pagestyle{fancy}
\delimitershortfall-1sp
\usepackage[fontsize=14pt]{scrextend}
\renewcommand{\baselinestretch}{1.5}
\setlength{\parindent}{1.25ex}

\newcommand{\anonsection}[1]{
    \phantomsection % Корректный переход по ссылкам в содержании
    \section*{#1}
    \addcontentsline{toc}{section}{\uppercase{#1}}
}

\addbibresource{references.bib}
\renewcommand\refname{Список использованных источников}

\begin{document}
\maketitle
\newpage

\subfile{annotation}
\newpage

\tableofcontents
\newpage

\section{Обозначения и основные определения}
\subfile{glossary}
\newpage

\section{Введение}
\subfile{introduction}
\newpage

\section{Материалы и методы}
\subfile{materials_and_methods}
\newpage  

\section{Полученные результаты}
\subfile{results}
\newpage

\section{Заключение и план дальнейших исследований}
\subfile{conclusions}
\newpage

\nocite{*}
\newpage

\pagestyle{bib} 
\section{Список использованных источников}
\printbibliography[heading=none]

\end{document}
